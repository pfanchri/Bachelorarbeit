% Header

\documentclass[
	a4paper,
	%oneside,
	twoside,
	12pt,
	numbers=noenddot,
	openright,	% Kapitel beginnt auf rechter Seite
	titlepage,	% Titelblatt auf eigener Seite
	bibliography=totoc,		% Literaturverzeichnis
	listof=totoc,	% Listen ins Inhaltsverzeichnis
	twoside,  % zweiseitig
]{scrbook}



\usepackage[english,ngerman,shorthands=off]{babel} % Deutsche Sprachanpassungen
\usepackage[T1]{fontenc}    % Silbentrennung bei Sonderzeichen
\usepackage[utf8]{inputenc} % Direkte Angabe von Umlauten im Dokument.
\usepackage{lmodern}					% Umlaute markierbar in PDF
\usepackage{textcomp}       % Zusätzliche Symbolzeichen
\usepackage{microtype}       % Optimierter Rand
\usepackage{amsmath}					% Formeln
\usepackage{graphicx}					% Graphiken einbinden
\usepackage{float}
\usepackage[german]{nomencl}
\usepackage[countmax]{subfloat}
\usepackage{wallpaper}

%\clubpenalty = 10000 % schliesst Schusterjungen aus
%\widowpenalty = 10000 % schliesst Hurenkinder aus

% Farben definieren für package listings 
\usepackage{color}
\definecolor{middlegray}{rgb}{0.5,0.5,0.5}
 \definecolor{lightgray}{rgb}{0.8,0.8,0.8}
 \definecolor{orange}{rgb}{0.8,0.3,0.3}
 \definecolor{yac}{rgb}{0.6,0.6,0.1}
 \definecolor{commentgreen}{rgb}{0.25,0.5,0.37}
 \definecolor{keyred}{RGB}{127,0,85}
 
% C-Quellcode einfügen
\usepackage{listings} 				
\lstset{
   basicstyle=\scriptsize\ttfamily,
   keywordstyle=\bfseries\ttfamily\color{keyred},
   stringstyle=\color{blue}\ttfamily,
   commentstyle=\color{commentgreen}\ttfamily,
   emph={square}, 
   emphstyle=\color{blue}\texttt,
   emph={[2]root,base},
   emphstyle={[2]\color{yac}\texttt},
   showstringspaces=false,
   flexiblecolumns=false,
   tabsize=2,
   numbers=left,
   numberstyle=\tiny,
   numberblanklines=false,
   stepnumber=1,
   numbersep=10pt,
   xleftmargin=15pt,
  breaklines=true,
  breakatwhitespace=false
 }


\setcounter{tocdepth}{3} %Gibt an, bis in welche Tiefe das Inhaltsverzeichnis geht
\setcounter{secnumdepth}{3}

\addto\captionsngerman{
\renewcommand{\figurename}{Abb.}
\renewcommand{\tablename}{Tab.}
} %Sorgt dafürm, dass statt Abbildung Abb. steht und statt Tabelle Tab.
 
\usepackage{titlesec}
\titleformat{\chapter}[display]
 {\bfseries\huge}
  {\filleft\Large\chaptertitlename~\thechapter}
  {3ex}
  {\titlerule\vspace{1.5ex}\filright}
  [\vspace{1ex}\titlerule]

\usepackage{acronym}
\usepackage{glossaries}
\usepackage{wrapfig}
\usepackage{units}
\usepackage{icomma} %Korrekte Abstände bei Dezimalkommas im Mathemodus

% % % %LITERATURVERWALTUNG
	% % % ACHTUNG: Zum kompilieren muss biber, nicht bibtex verwendet werden!
	\usepackage[
		backend=biber,
		style=ieee,
		backref=true,%Referenzen im LieVerzeichnis auf verwendete Seiten im dokument
		isbn=false,
		doi=false,
		eprint=false,
		url=false,
		safeinputenc,
		defernumbers=true]{biblatex}
	\usepackage{csquotes}
	
	%\ExecuteBibliographyOptions{sorting=ydmdnt}%Sortierung, none bedeutet nach Vorkommen
	\ExecuteBibliographyOptions{firstinits=true} %Vor- und Mittelnamen als Initialen ausgeben
	\ExecuteBibliographyOptions{maxbibnames=6} %max angezeigte namen, dann auf minnames reduziert und u.a. ergänzt
	\ExecuteBibliographyOptions{minbibnames=4} %siehe maxnames
	\ExecuteBibliographyOptions{maxcitenames=3} %analog für citeauthor, o.ä.
	\ExecuteBibliographyOptions{mincitenames=1} %siehe maxnames

% %Tikz für schöne Zeichnungen:
	% Tikz pakete & optionen
	\usepackage{tikz}
	\usepackage[european,americaninductors]{circuitikz}
	\usetikzlibrary{patterns}
	% % Externalize settings
	\usetikzlibrary{external} 
	\tikzset{external/up to date check=simple}
	%\tikzsetexternalprefix{} %path to graphics ACHTUNG: In header.tex definiert!
	\tikzexternalize%[shell escape=-enable-write18] %option nötig für miktex
	\pgfkeys{/pgf/images/include external/.code={\href{file:#1}{\pgfimage{#1}}}} %Links in pdf zu externen PDF-Bildern für Debug

	% pgfplot pakete & Optionen
	\usepackage{pgfplots}
	
	\usepgfplotslibrary{groupplots}
	
	\pgfplotsset{compat=1.10}
	\pgfplotsset{every tick label/.style={font={\scriptsize\sansmath\boldmath}}}
	\pgfplotsset{every axis label/.style={font={\boldmath\small\sffamily\bfseries}}}
	\pgfplotsset{every legend /.style={font={\boldmath\small\sffamily\bfseries}}}
	\pgfplotsset{every label/.style={font={\boldmath\small\sffamily\bfseries}}}
	
	\pgfplotsset{width=\textwidth}
	\pgfplotsset{grid=both}
	\pgfplotsset{major tick style={thin,black}}% modifies the style ‘every tick’
	\pgfplotsset{minor tick style={very thin,black}}% modifies the style ‘every tick’
	\pgfplotsset{major grid style={thin}} %modifies the style ‘every major grid’
	\pgfplotsset{minor grid style={very thin}} % modifies the style ‘every minor tick’
	\pgfplotsset{minor x tick num=1} %n minor ticks zwischen major ticks(major tiks müssen selben abstand haben)
	\pgfplotsset{minor y tick num=1} %n minor ticks zwischen major ticks(major tiks müssen selben abstand haben)
	\pgfplotsset{xlabel near ticks,ylabel near ticks}
	%For ylabel offsets:
	
	%For ylabel offsets:
	\pgfplotsset{ylabsh/.style={every axis y label/.style={at={(0,0.5)}, xshift=#1, rotate=90}}}
	\pgfplotsset{xlabsh/.style={every axis x label/.style={at={(0.5,0)}, yshift=#1}}}
	%Erweiterung Each nth point mehr als 100 Punkte auslassen
	\makeatletter
	\pgfplotsset{
	/pgfplots/each nth point**/.style={%
	/pgfplots/x filter/.append code={%
	        \ifnum\coordindex=0
	                \def\c@pgfplots@eachnthpoint@xfilter{0}%
	                \edef\c@pgfplots@eachnthpoint@xfilter@cmp{#1}%
	        \else
	                \pgfplotsutil@advancestringcounter\c@pgfplots@eachnthpoint@xfilter
	                \ifx\c@pgfplots@eachnthpoint@xfilter@cmp\c@pgfplots@eachnthpoint@xfilter
	                        \def\c@pgfplots@eachnthpoint@xfilter{0}%
	                \else
	                        \let\pgfmathresult\pgfutil@empty
	                \fi
	        \fi
	}%
	},
	}
	\makeatother
	
	%cycle list für plots, !!!! \addplot+ verwenden!
	\pgfplotsset{
	    cycle list={
	        blue,
	        red,
	        {black, thick, dashed},
	        violet
	    }
	}
	
	



%%%% Abkürzungsverzeichnis %%%%
	%\usepackage[intoc]{nomencl}						% Abkürzungsverzeichnis
	%\usepackage{nomencl}						% Abkürzungsverzeichnis
	%\renewcommand{\nomname}{Abkürzungsverzeichnis}	% Deutsche Überschrift
	%\setlength{\nomlabelwidth}{.25\hsize}			% Punkte zw. Abkürzung und Erklärung
	%\renewcommand{\nomlabel}[1]{#1 \dotfill}
	%\setlength{\nomitemsep}{-\parsep}				% Zeilenabstände verkleinern
	%\makenomenclature
	\usepackage{geometry}
	\geometry{a4paper, top=30mm, left=26mm, right=26mm, bottom=35mm,
	headsep=10mm, footskip=12mm} %Legt Seitenränder fest


%%%% Kopf- und Fußzeilen %%%%
	%\usepackage[automark, headsepline, footsepline]{scrpage2}
	\usepackage[automark, headsepline]{scrpage2}		% Kopf- und Fußzeilen
	\pagestyle{scrheadings}
	\clearscrheadfoot
	\ohead{\headmark}
	\ihead{Bachelorarbeit}
	\cfoot{Stand: \today }
	%\cfoot[\pagemark]{\pagemark}
	%\ohead[\headmark]{\headmark}
	%\ihead[Masterarbeit]{Masterarbeit}
	\ofoot{\pagemark}
	\ofoot[\pagemark]{\pagemark}
	%\ifoot{Lehrstuhl für Technische Elektronik}
	%\ifoot[Lehrstuhl für Technische Elektronik]{Lehrstuhl für Technische Elektronik}


% Hyperref als letztes laden
	\usepackage[pdftex,hidelinks]{hyperref} %Für Links im PDF
	\hypersetup{
	  pdftitle    = {Abschlussarbeit Christof Pfannenmüller},
	  pdfsubject  = {Thesis},
	  pdfauthor   = {\authorname},
	  pdfkeywords = {Bachelorarbeit, Pfannenmüller,Basisstation,TDA,XMC4500,Feldstärke,Ortung} ,
	}
%pfade definieren
	\graphicspath{{abbildungen/}}
	\tikzsetexternalprefix{abbildungen/tikz-ext-out/\thesubsection-} %path to graphics