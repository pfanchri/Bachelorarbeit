\chapter{Feldtest}
\label{sec:Feldtest}
\pagestyle{scrheadings}
\section{Aufbau}
Zur Evaluation der Basisstation wurde ein  XMC4500 Relax Kit von Infineon mit einem aufgesteckten Evaluations-Board für den TDA5340 betrieben. Mit Hilfe einer Powerbank konnte dieses mobil über den \ac{USB}-Anschluss des Relax Kit betrieben werden. %genaue werte eingeben und rechnung aus exekl sheet
Dieses wurde auf eine Sendefrequenz von $...$ und eine Empfangsfrequenz von $...$ programmiert was durch die Werte für die \ac{PLL} im TDA5340 eingestellt wurde. 
Die Basisistation wurde mit sechs Antennen, die einen Verstärkungsfaktor von $3,6$dBi und eine Mittenfrequenz von $868$MHz aufwiesen, bestückt. Die Basis wurde über \ac{USB} an den Computer zur auswertung angeschlossen. Das Auslesen der durch virtuellen COM-Port übertragenen Daten erfolgte mit  PuTTY. Die Messungen fanden innerhalb des Gebäudes statt.
\section{Durchführung}
Es wurden  im selben Raum von diversen Positionen durch einen Tastendruck am Relax Kit eine Übertragung ausgelöst. Dabei wurden die zuvor einprogrammierte Zeichenkette $1,2,3,4,5,6,7,8,9$ ausgesendet. Der Abstand zur Basisstation betrug im ersten Test  $3,30$m und wurde nach jeder Übertragung um $30$cm verringert. 
In einem zweiten Test wurde ebenfalls mit einem Abstand von $3,3$m gestartet.  Nach jeder Übertragung wurde der Sender $30$cm  von der Startposition aus, entlang einer  Linie, rechtwinklig zur Sichtverbindung Startpunkt-Basis, vom Relax Kit entfernt. Die gemessenen Werte wurden zur weiteren Auswertung abgespeichert.
In beiden Test war die Basisstation so ausgerichtet, das TDA1 in Richtung der gemeinsamen Startposition der Tests zeigte. Logic Analyser und Debugger waren während der Tests nicht an der Basisstation angeschlossen. So sollten Abschattungseffekte durch dieser verhindert werden. Die Basisstation wurde auf einem $70$cm hohen Tisch aufgestellt. Der Sender wurde auf gleicher Höhe freischwebend bewegt. Es wurden in beiden Tests zehn Messpunkte gesendet. Sowohl die Antennen an der Basisstation, als auch jene am Relax Kit waren senkrecht nach oben zeigend ausgerichtet.
\section{Ergebnisse und Auswertung}
Auffallend ist, das zwar in jedem Test zehn mal durch das Drücken des Tasters eine Übertragung ausgelöst wurde, jedoch öfter eine Übertragung an der Basistation registriert wurde. Im Ersten Feldtest wurden fünfzehn, im zweiten sogar sechzehn gültige Übertragungen von der Basisstation an den Hostcomputer ausgegeben. 
nicht verändert->evtl nicht empfangen

zeitstempel einfügen um doppelte übertragungen zu erkennen
