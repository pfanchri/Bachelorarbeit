\chapter{Feldtest}
\label{sec:Feldtest}
\pagestyle{scrheadings}
\section{Aufbau}
Zur Evaluation der Basisstation wurde ein  XMC4500 Relax Kit von Infineon mit einem aufgesteckten Evaluations-Board für den TDA5340 betrieben. Mit Hilfe einer Powerbank konnte dieses mobil über den \ac{USB}-Anschluss des Relax Kit betrieben werden. %genaue werte eingeben und rechnung aus exekl sheet
Dieses wurde auf eine Sendefrequenz von $...$ und eine Empfangsfrequenz von $...$ programmiert was durch die Werte für die \ac{PLL} im TDA5340 eingestellt wurde. 
Die Basisistation wurde mit sechs Antennen, die einen Verstärkungsfaktor von $3,6$dBi und eine Mittenfrequenz von $868$MHz aufwiesen, bestückt. Die Basis wurde über \ac{USB} an den Computer zur auswertung angeschlossen. Das Auslesen der durch virtuellen COM-Port übertragenen Daten erfolgte mit  PuTTY. Die Messungen fanden innerhalb des Gebäudes statt.
\section{Durchführung}
Es wurden  im selben Raum von diversen Positionen durch einen Tastendruck am Relax Kit eine Übertragung ausgelöst. Dabei wurden die zuvor einprogrammierte Zeichenkette $1,2,3,4,5,6,7,8,9$ ausgesendet. Der Abstand zur Basisstation betrug im ersten Test  $3,30$m und wurde nach jeder Übertragung um $30$cm verringert. 
In einem zweiten Test wurde ebenfalls mit einem Abstand von $3,3$m gestartet.  Nach jeder Übertragung wurde der Sender $30$cm  von der Startposition aus, entlang einer  Linie, rechtwinklig zur Sichtverbindung Startpunkt-Basis, vom Relax Kit entfernt. Die gemessenen Werte wurden zur weiteren Auswertung abgespeichert.
In beiden Test war die Basisstation so ausgerichtet, das TDA1 in Richtung der gemeinsamen Startposition der Tests zeigte. Logic Analyser und Debugger waren während der Tests nicht an der Basisstation angeschlossen. So sollten Abschattungseffekte durch dieser verhindert werden. Die Basisstation wurde auf einem $70$cm hohen Tisch aufgestellt. Der Sender wurde auf gleicher Höhe freischwebend bewegt. Es wurden in beiden Tests zehn Messpunkte gesendet. Sowohl die Antennen an der Basisstation, als auch jene am Relax Kit waren senkrecht nach oben zeigend ausgerichtet.
\section{Ergebnisse und Auswertung}
Auffallend ist, das zwar in jedem Test zehn mal durch das Drücken des Tasters eine Übertragung ausgelöst wurde, jedoch öfter eine Übertragung an der Basistation registriert wurde. Im Ersten Feldtest wurden fünfzehn, im zweiten sogar sechzehn gültige Übertragungen von der Basisstation an den Hostcomputer ausgegeben. 
Im ersten Test konnten bei der ersten Übertragung an den Transceivern eins bis vier keine Daten empfangen  werden. Erst in der darauffolgenden zweiten erkannten Übertragung wurde hier die gesendete Zahlenfolge empfangen. Bei der zweiten Übertragung, welche über die Konsole ausgegeben wurde, stimmten die Werte von TDA5 und TDA6 in allen ausgelesenen Registern mit den Messwerten der ersten Übertragung überein.  Daraus ist zu folgern, das es sich bei diesen Werten noch um die Messungen aus dem ersten Transfer handelt. Somit wäre zu folgern, das durch eine leichte Verzögerung zwischen den TDA5340 der steuernde Mikrocontroller eine gemeinsame Übermittlung als zwei getrennte Übertragungen interpretiert hat.
Transceiver 1 konnte in dem Test erst ab der vierten Übertragung gültige Daten empfangen. Außerdem waren die gemessenen Empfangsleistungen stets geringer als $-100$dBm, lediglich  bei der letzten Übertragung, welche bei einem Abstand von $30$cm stattfand, konnte hier ein Wert von $-97$dBm gemessen werden. Da diese  sehr schwachen Empfangsleistungen in vorherigen Tests nicht auftraten ist zu vermuten, das etwa eine nicht richtig verbundene Antenne  Grund des schwachen Empfangswertes war.
TDA3 konnte in keiner einzigen Übertragung passable Messwerte liefern. Die errechnete und ausgegebene Empfangsfeldstärke von $-114$dBm entsprach dem minimal möglichen Ausgabewert der dafür verwendeten Funktion. Es ist also davon auszugehen, das in diesem Transceiver nie eine Funkverbindung erkannt wurde. Gründe dafür wären ein Fehler im Anpassnetzwerk zwischen der Antenne und dem Transceiver oder ein Defekt des selbigen. Letzteres ist eher unwahrscheinlich, da eine Verbindung über \ac{SPI} mit dem \ac{IC} möglich war. Lediglich ein Teildefekt  in der RF-Sektion des Chips wäre also denkbar.
Daneben waren vereinzelt auch noch Übertragungen zu erkennen, in denen Registerwerte mit den Messungen aus den folgenden oder vorherigen Übertragungen übereinstimmten. Auch hier ist zu vermuten, das einzelne Transceiver eine Übertragung nicht erkennen konnten.
