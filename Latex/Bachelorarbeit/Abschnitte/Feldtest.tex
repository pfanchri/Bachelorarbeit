\chapter{Feldtest}
\label{sec:Feldtest}
\pagestyle{scrheadings}
\section{Aufbau}
Zur Evaluation der Basisstation wurde ein  XMC4500 Relax Kit von Infineon mit einem aufgesteckten Evaluations-Board für den TDA5340 betrieben. Mit Hilfe einer Powerbank konnte dieses mobil über den \ac{USB}-Anschluss des Relax Kit betrieben werden. %genaue werte eingeben und rechnung aus exekl sheet
Dieses wurde auf eine Sendefrequenz von $...$ und eine Empfangsfrequenz von $...$ programmiert was durch die Werte für die \ac{PLL} im TDA5340 eingestellt wurde. 
Die Basisistation wurde mit sechs Antennen, die einen Verstärkungsfaktor von $3,6$dBi und eine Mittenfrequenz von $868$MHz aufwiesen, bestückt. Die Basis wurde über \ac{USB} an den Computer zur auswertung angeschlossen. Das Auslesen der durch virtuellen COM-Port übertragenen Daten erfolgte mit  PuTTY. Die Messungen fanden innerhalb des Gebäudes statt.
\section{Durchführung}
Es wurden  im selben Raum von diversen Positionen durch einen Tastendruck am Relax Kit eine Übertragung ausgelöst. Dabei wurden die zuvor einprogrammierte Zeichenkette $1,2,3,4,5,6,7,8,9$ ausgesendet. Der Abstand zur Basisstation betrug im ersten Test etwa $...$m und wurde nach jeder Übertragung um $...$cm verringert.
In einem zweiten Test wurde ebenfalls mit einem Abstand von $...$m gestartet.  Nach jeder Übertragung wurde der Sender $...$cm  von der Startposition aus, entlang einer  Linie, rechtwinklig zur Sichtverbindung Startpunkt-Basis, vom Relax Kit entfernt. Die gemessenen Werte wurden zur weiteren Auswertung abgespeichert.
\section{Ergebnisse und Auswertung}
nicht verändert->evtl nicht empfangen