% Kurzfassung

\chapter*{Kurzfassung}
\label{sec:kurzfassung}
\pagestyle{scrheadings}

Ziel der vorliegenden Arbeit war der Aufbau einer Basisstation mit sechs unabhängigen Transceivern. Der Einsatz dieses Aufbaus diente der späteren Lokalisierung von Sensoren oder anderen Sendern im Sub-GHz-Frequenzbereich um \unit[868]{MHz}. Die relative Ortsbestimmung zur Basis sollte energieeffizient sein und gleichzeitig eine hohe Auflösung bieten. 

Die genaue Position des Senders wurde dazu auf Basis der unterschiedlichen Feldstärken an den Transceivern eruiert. Ausgenutzt wurde dabei, dass typische integrierte Transceiver die Empfangsfeldstärke selbst auswerten und bereitstellen, sodass diese ausgelesen werden kann. Das Erkennen von Übertragungen und das anschließende Auslesen der anfallenden Daten wurde dabei von einem Mikrocontroller übernommen. Zur weiteren Verarbeitung der Daten sollten diese anschließend einem Computer zur Verfügung gestellt werden. Dazu wurde sowohl eine USB- als auch eine Netzwerk-Schnittstelle vorgesehen. Die beim Aufbau der Platine verwendete Hardware basierte zum Großteil auf Bauteilen des Herstellers Infineon. 
Beim Layout der Platine wurden die sechs Transceiver sternförmig und regelmäßig um den Mikrocontroller und dessen Peripherie angeordnet, um ein gleichmäßiges Empfangsverhalten aus allen Raumrichtungen zu gewährleisten. Die Funksegmente der Platine wurden so gestaltet, dass diese bei Bedarf abgetrennt und mit einer Kabelverbindung weiter voneinander entfernt werden konnten. Antennen zum Senden wurden über Steckverbinder an die Basisstation angeschlossen.

Der anschließende Softwareentwurf für die Basisstation nutzte zu einem Großteil bereits bestehende Bibliotheken und hatte zur Aufgabe, ankommende Übertragungen zu erkennen, die zur Ortung notwendigen gemessenen Werte abzufragen und an den Hostcomputer weiterzuleiten. 

