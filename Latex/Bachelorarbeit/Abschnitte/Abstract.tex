% Abstract
\newpage
\chapter*{Abstract}
\label{sec:abstract}
\pagestyle{scrheadings}

Current localization measurements  by radio frequency have been complex and consume plenty of energy. However almost every Receiver has knowledge of the electrical field strength, correlating to distance from Transmitter, this information is nearly unused. A multi Transceiver base station should start communication with a mobile wireless sensor. The relative positioning to the base could be calculated by the received signal strength (RSSI) already provided from Transceivers without additional components. Due to the permeability of walls, the possible range and the wavelength, associated to resolution of localization,   Sub-GHz frequency range is used.  Therefoer six identical Transceiver-ICs were arranged over all horizontal directions in space. By use of focused antenna the recognition of transmission direction could be corrected. Coordination of the ICs would be realized by a XMC4500 \textmu C connected to the Transceivers with SPI and IRQ line for finished transmission. Distribution of received data and signal strength measurements to a host computer is accomplished by the XMC4500 over Ethernet and USB.