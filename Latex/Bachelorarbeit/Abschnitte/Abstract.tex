% Abstract
\newpage
\chapter*{Abstract}
\label{sec:abstract}
\pagestyle{scrheadings}

Zur Lokalisierung von mobilen Sensorknoten, welche in einem Sub-GHz-Frequenzbereich von 868 MHz arbeiten, sollte eine energieeffiziente Art der Ortung umgesetzt werden. Dazu sollte mithilfe einer auf Feldstärke basierten Ortung die genaue Position der Sensorknoten festgestellt werden. Um dies zu ermöglichen wurde eine Basisstation mit sechs Transceiver-ICs der Bauart TDA5340 entworfen. Die Steuerung übernahm ein Mikrocontroller der Baureihe XMC4500. Diese Basisstation sollte dabei ein Auslesen der empfangen Daten  sowohl über den USB-Standard, als auch über Ethernet als zweite Kommunikationsschnittstelle ermöglichen. Die dabei verwendete Hardware basierte zum Großteil auf Bauteilen des Herstellers Infineon. 
Die Platine der Basisstation wurde mit Altium Designer entwickelt und umgesetzt. Dabei wurde die  Verbindung zwischen der Steuereinheit und den Transceivern mit dem SPI-Protokoll umgesetzt. Die einzelnen Sende-/Empfangseinheiten wurden dabei gleichmäßig in alle Raumrichtungen zeigend angeordnet und so gestaltet, das diese bei bedarf abgetrennt und mit einer Kabelverbindung weiter voneinander entfernt konnten.
Die Peripherie der verwendeten TDA5340 Transceiver generierte im Programmablauf nach einer Kommunikation mit dem Sensor ein Interrupt Signal. Dies erlaubte dem Mikrocontroller die Daten der einzelnen Empfangseinheiten auszulesen, zu speichern und zu einem späteren Zeitpunkt weiterzuleiten. Beim Messen der Feldstärke wurde ausgenutzt, dass die vorliegende Feldstärke bereits durch den TDA5340-Empfänger zur Verfügung gestellt wurde.


Antenne in verschiedene Richtungen ...