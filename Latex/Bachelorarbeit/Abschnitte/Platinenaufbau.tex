\chapter{Platinenaufbau}
\label{sec:Platinenaufbau}
\pagestyle{scrheadings}

\section{Vorüberlegungen}
Um Sicherzustellen, das alle Antennen gleichmäßig in die sechs vorgegebenen Raumrichtungen abstrahlen sollte bereits die Platine symmetrisch aufgebaut werden. Dazu wurden zuerst das Layout der sechs identischen Transceiver-Einheiten mit dem TDA5340 Baustein und den Antennen erstellt und anschließend gleichmäßig um die weiten notwendigen notwendigen Segmente der Schaltung angeordnet. 
\section{Layoutprogramm Altium Designer}
Bei dem Entwicklungswerkzeug \enquote{Altium Designer} des Entwicklers Altium Limited handelt es sich um ein System zum Entwurf von  gedruckten  Schaltungen oder PCBs (Printed Circuit Boards). Ein solches Programm wird auch als Electronic Design Automation (EDA) oder ECAD für electronic CAD bezeichnet, da es den Entwickler bei der Umsetzung der Anforderungen in einen Schaltplan unterstützen soll.
Wie viele andere EDA-Programme ist auch Altium Designer so aufgebaut das sich der Entwickler zuerst mit dem allgemeinen Schaltplan befassen kann und erst zu einem späteren Zeitpunkt die tatsächliche Anordnung auf dem PCB-Substrat festgelegt wird. Somit können zuerst im Schematic Editor die Funktionen der Schaltung umgesetzt werden. Dazu werden die verwendeten Bauteile aus zuvor angelegten Bibliotheken verwendet oder es werden bestehende Bibliotheken verwendete, die etwa vom Hersteller der Bauteile zur Verfügung gestellt werden. Altium selbst bietet hierfür auch diverse Möglichkeiten an und stellt Bauteile nach Hersteller und Art geordnet bereit.
In den Bibliotheken sind alle im weiteren Verlauf benötigten Informationen über die einzelnen Bauteile enthalten. So liegen dort etwa  entsprechenden Abbildungen für das Bauteil im  Schaltplan  vor. In den so genannten \enquote{Footprints} zu jedem Bauteil, welche ebenfalls in den Bibliotheken enthalten sind, wurde zuvor die für das physikalische Gehäuse notwendigen Abmessungen und Lötpads festgelegt. Da es Bauteile wie den verwendeten  Mikrocontroller in verschiedenen Gehäusen geben kann besteht somit auch die Möglichkeit hier verschiedene Footprints zu wählen. Da viele Gehäuse herstellerübergreifend genormt sind, konnten teilweise bestehende Footprints genutzt werden oder diese mehrfach verwendet werden.
Ebenfalls 

Bauteileingabe und schaltplan
MUlti channel - multi sheet





Altium Designer ist dabei in drei Teilbereiche unterteilt: im \enquote{Board Planning Mode}  liegt der Fokus auf dem Anordnen der einzelnen Bauteile und Komponenten auf der Leiterplatte, außerdem wird in diesem Bereich die Form und Ausmaße der Leiterplatte festgelegt. Im 2D-Modus des PCB-Editor lassen sich anschließend die aus der Definition im Schaltplan ergebenden elektrischen Verbindungen örtlich auf den verschiedenen Kupferebenen (Layern) anordnen. Die Hauptarbeit findet also in diesem Teil des PCB-Editors statt. Der 3D-Modus dient anschließend zur Evaluation des Designs und zur Anpassung an Gehäuse oder andere Komponenten.
DRC

\section{verwendete Hardware}

\subsection{TDA5340}
Der verwendete Transceiver TDA5340 wird von Infineon Technologies AG  entwickelt und vertrieben. Er ist teil der SmartLEWIS\textsuperscript{TM} Produktfamilie die  energiesparende Lösungen im Frequenzspektrum unterhalb von einem Gigaherz bietet. 
\subsection{XMC4500}

\subsection{Ethernet}
Die Ethernetschnittstelle der Basistation basiert auf dem RelaxKit von Infineon. Genau wie im Evaluations Board des Herstellers wurde der Ethernet-Controller KSZ8031RNL von Mircel inc. verwendet. Dieser stellt alle wichtigen Peripherien selbst zur Verfügung und muss somit nur noch durch ein Quarz und diverse Kapazitäten und Induktivitäten an den Versorgungsleitungen ergänzt werden. Da die im Controller verbaute Stufe zur Interruptgenerierung nur über einen schwachen Pull-Up Widerstand verfügt, musste ein externer Widerstand von $1k\Omega$ verbaut werden. Am Reset-Eingang wurde ebenfalls ein Pull-Up Widerstand verbaut. Dieser wurde um zwei Dioden sowie einen Kondensator zu der im Datenblatt empfohlenen Verschaltung erweitert. So kann sichergestellt werden, das sowohl beim Anlegen einer Spannung an das Gesamtsystem, als auch bei einem Reset des Ethernetbausteins durch den steuernden Mikrocontroller alle Spannungen im sicheren Bereich liegen und die Funktion gewähleistet ist.

\subsection{Spannungsversorgung}
\section{Generierte Dokumente}