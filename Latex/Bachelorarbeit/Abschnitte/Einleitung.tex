
\chapter{Einleitung}
\label{sec:Einleitung}
\pagestyle{scrheadings}



\section{Motivation}
Die Lokalisierung von Objekten bildet in der Sicherheits- und Automatisierungstechnik eine immer größere Basisdisziplin. Sie ist daher seit Anbeginn des Informationszeitalters eine zentrale Aufgabe der Elektronik. Im Zeitalter von Industrie 4.0 und allgegenwärtigen autonomen Systemen wird die Ortung von Objekten unersetzlich.


Bisherige Ansätze zur elektronischen Ortsbestimmung sind meist aufwändig mit einem hohen Energieverbrauch und benötigen viele Teilkomponenten. Dies macht die elektronische Ortsbestimmung teuer, was sich vor allem im Betrieb von Systemen wie GPS oder Galileo zeigt. Daneben haben diese Systeme vor allem den Nachteil, dass die verwendeten Frequenzen Wände kaum durchdringen und somit innerhalb von Gebäuden nicht verwendet werden können. Genau dieses Anwendungsszenario stellt jedoch in der Industrie 4.0 eine typische Fabrik dar. Dies führt dazu, dass für das am häufigsten verwendete Beispiel, einen autonom agierenden Roboter in einer Lagerhalle, andere Techniken zur Ortsbestimmung eingesetzt werden müssen. 

Ortungsansätze für ein solches autonomes bewegliches System existieren bereits und werden wie bei der WLAN-basierten Ortung auch eingesetzt. Diese Systeme haben jedoch den Nachteil, dass sie viele Sender mit bekanntem Ort benötigen. Dies erfordert eine aufwändige Kalibrierung des Systems und führt vor allem zu einem großen Energieverbrauch des Gesamtsystems, da alle stationären Einheiten dauerhaft senden müssen, um eine Ortung zu ermöglichen.
Ein weiterer Nachteil der Systeme ist, dass nur der zu ortende Client die Informationen über seinen Standort hat. Für ein außenstehendes System, wie den steuernden Hauptrechner einer Industrieanlage, ist nicht festzustellen, wo das Objekt sich befindet. 
Durch diese Probleme ließe sich das Anwendungsszenario eines einzelnen autarken Funksensors, dessen Messwert vom Hauptrechner abgefragt wird, nur schwer realisieren. Der Sensor müsste über eine große Energiereserve verfügen und ständig aktiv sein, um zu erkennen wenn er dazu aufgefordert wird seine Position zu ermitteln und mitzuteilen.
Durch das Verlagern der Ortungsaufgabe auf das stationäre System müsste der Sensor nur beim Vorliegen eines neuen Messwertes eine Funkverbindung aufbauen. Die Basisstation könnte aus dem empfangenen Signal sowohl den Messwert extrahieren als auch eine Lokalisierung durchführen.
Sollte der stationäre Teil  aus nur einer Einheit bestehen, wäre auch eine Änderung am Gesamtsystems, wie eine Ortsänderung der stationären Einheit, einfach möglich, da nur die relative Position zu dieser ermittelt wird, weshalb eine Neukalibrierung des Systems entfällt. Der Sensor selbst benötigt im Normalfall keine Information über seinen Aufenthaltsort. Dies erlaubt eine mobile Ortung da die Basis unkompliziert bewegt werden kann.


Es bietet sich hierfür eine Nutzung des Sub-GHz-Frequenzbandes an, welches deutlich weniger belegt ist, in dem meist nur kurze Funkdauern verwendet werden und in welchem eine Übertragung vor allem energieeffizienter möglich ist \cite{SabolcikGHzoderSub}.
Ein weiterer Vorteil des verwendeten Frequenzbandes sind die bessere Penetrationseigenschaften durch Wände und Personen im Vergleich zu Ansätzen, die etwa auf WLAN basieren.


\section{Zieldefinition}
Zum Umsetzen einer solchen oben beschriebenen Ortung sollte eine Basisstation aufgebaut werden, welche als stationäre Einheit einen Sender orten könnte.
Die Aufgabenstellung dieser Bachelorarbeit bestand im Einarbeiten in den vorgegeben Transceiver TDA5340 und den Mikrocontroller XMC4500. Bei der Gestaltung des Layouts für die Platine mit Altium Designer wurden diese über einen SPI-Bus verbunden sein. Die Platine sollte zusätzlich ein Konzept zur Bereitstellung der Versorgungsspannung enthalten. Um die Weiterverarbeitung der bezogenen Messwerte sicherzustellen, war eine LAN-Schnittstelle sowie eine Möglichkeit für einen USB-Anschluss zu einem PC zu integrieren.

Das ursprünglich geplante Design von PCB-Antennen an den Transceivern auf der Platine wurde wegen des Umfangs der dazu  notwendigen Simulationen  zu Beginn aus der Aufgabenstellung entfernt. Stattdessen wurden zugekaufte Antennen mit der Basisstation verwendet, welche über Steckverbinder angeschlossen wurden.

Eine funktionierende Firmware für das Initialisieren und Betreiben der Basisstation war ebenfalls mit der dazu notwendigen Entwicklungsumgebung zu erstellen. Dabei musste der Mikrocontroller auf, vom Transceiver ausgelösten, Interrupts für ankommende Übertragungen reagieren und Messwerte erfassen beides wurde anschließend entsprechend weitergeleitet. Durch die Verwendung von Makros beim Softwareentwurf sollte eine spätere Anpassung, wie das Tauschen von Pins leichter möglich sein.


\section{Projektmanagement}
Die vorliegende Arbeit wurde innerhalb von fünf Monaten am Lehrstuhl für Technische Elektronik der Universität Erlangen-Nürnberg angefertigt. Dabei lag der Fokus in den ersten beiden Monaten auf dem Layout der Platine und dem anschließenden Bestücken. In der folgenden Zeit wurde vermehrt auf die Software für den Betrieb der Basisstation eingegangen. Außerdem wurde in den letzten zwei Monaten die Dokumentation mit Latex erstellt. Als Versionskontrolle für das gesamte Projekt wurde Github eingesetzt.

%todo: evtl erweitern und mehr hinzufügen