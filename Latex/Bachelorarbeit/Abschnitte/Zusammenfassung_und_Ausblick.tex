\chapter{Zusammenfassung und Ausblick}
\label{sec:Zusammenfassung}
\pagestyle{scrheadings}




Zur weiteren Verbesserung wäre noch das Eruieren des Grundes für den schlechten Empfang am Transceiver $3$ notwendig. So könnte durch das Anschließen eines Signalgenerators an die Antennenbuchse der entsprechenden Transceiverbaugruppe ein möglicher Fehler im Anpassnetzwerk aufgezeigt werden. Sollte auch dies nicht zu einer Verbesserung führen, wäre ein Austausch des \acp{IC} notwendig.
Für das bessere Erkennen von doppelten Übertragungen wäre das Einfügen eines Zeitstempel in die Ausgabe hilfreich. Dadurch könnten Übertragungen die kurz hintereinander eintreffen markiert und entsprechend zu einer korrekten zusammengefügt werden. Zu diesem Zweck würde es sich anbieten die Realtime-Clock des XMC4500 zu verweden. Dazu würde das auf der Platine vorsorglich verbaute Uhrenquarz verwendet werden. 


Auch wäre für eine Veränderung an der Platine die Auswahl eines anderen Eingangspins am XMC für das vom PP2 Pin des Transceiver 6 kommenden Interruptsignals sinnvoll. So wären die TDA3 und TDA6 nicht an den selben Interruptkanal des Mikrocontroller angeschlossen. Die dazu notwenigen Änderungen an der Software würden sich auf die Änderungen der entsprechenden Makros in der Headerdatei beschränken.
Alternativ ließe sich möglicherweise das Problem der konkurrierenden Interrupts über Anpassungen in der Software lösen. So wäre es möglich das Interruptsignal einzelner TDA5340 nicht über den PP2 Pin auszugeben, sonder auch über die ebenfalls mit dem XMC verbundenen PP0 und PP1 Pins. Somit wäre ein verteilen auf einzelne Kanäle der \ac{ERU} wahrscheinlich möglich.



